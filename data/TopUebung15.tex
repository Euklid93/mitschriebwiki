\documentclass{article}
\usepackage[utf8]{inputenc}
\usepackage{mathrsfs}
\usepackage{stmaryrd}

\usepackage{mathe}
\usepackage{enumerate}
\usepackage{amscd}

\title{15. Topologie-Übung}
\author{Joachim Breitner}
\date{13. Februar 2008}

\begin{document}
\maketitle

\section*{Aufgabe 1}


Sei $f \in \mathbb C[X]$ nicht konstant, $W\da \{ f(z) \mid z\in \mathbb C, f'(z)=0\}$.

\paragraph{Behauptung:} $p:\{(z,w) \mid f(z)=w, W\notin W\} \to \mathbb C\setminus W$, $(z,w)\mapsto w$ ist eine Überlagerung.

Sei $w\in \mathbb C \setminus W$. Zu zeigen ist: Es gibt eine Umgebung $U\subseteq \mathbb C\setminus W$ von $w$: $p^{-1}(U) = \bigcup_{i\in I}V_i$, $V_i$ paarweise disjunkt, $V_i\cong U$.

Anschaulich: Sei $U$ eine $\epsilon$-Umgebung von $w$, die ganz in $\mathbb C\setminus W$ liegt, dann ist $p^{-1}(U)=\{(z,f(z))\mid f(z)\in U\}$

Sei $w\in \mathbb C\setminus W$ und $p^{-1}(w)=\{z_1,\ldots,z_n\}$, wobei $f'(z_i) \ne 0$, $i=1,\ldots,n$. Nach dem Umkehrsatz gilt dann: Für alle $i=1,\ldots,n$ gibt es eine offene Umgebung $U_i$ von $z_i$, so dass $f|_{U_i}$ bijektiv ist.

Wähle $\varepsilon$ so klein, dass für $i=1,\ldots,n$ gilt: $B_\varepsilon(z_i) \subseteq U$ und $B_\varepsilon(z_j)$ sind disjunkt für $i\ne j$ und setze
\[
U \da \bigcap_{i=1}^n f(B_\varepsilon(z_i))
\]
Es ist 
\[
p^{-1}(V)\da
\underbrace{\{(z,f(z)) \mid z\in (f|_{U_1})^{-1}(V)\}}_{\ad V_1}
\cup \cdots \cup
\underbrace{\{(z,f(z)) \mid z\in (f|_{U_n})^{-1}(V)\}}_{\ad V_n}
\]

Die $V_i$ sind nach Konstruktion offen, disjunkt und alle homöomorph zu $U$.

\section*{Aufgabe 2}
 Sei $X\da \bigcup_{n\in \mathbb N}K_{\frac 1n}( (\frac1n,0) )$, „Havaiianische Ohrringe“.

\paragraph{Vorüberlegung:} 
Ist $p:\tilde X\to X$ eine universelle Überlagerung, dann ist $X$ semi-lokal einfach zusammenhängend.

Denn: Ist $x\in X$ und $y\in p^{-1}(x)$, dann gibt es eine Umgebung $U$ von $x$ mit $p^{-1}(U)=\bigcup{i\in I} V_i$, wobei dei $V_i$ offen, paarweise disjunkt und zusammenhängend sind. Sei $V\da V_i$ für das $i$, für das gilt: $y\in V_i$, dann gibt es einen Homöomorphismus $q\da p|_V :V\to U$. Wir erhalten das kommutative Diagram:
\[
\begin{CD}
\pi_1(V,y) @>\pi_1(\iota)>> \pi_1(\tilde X, y) \\
@V\pi_1(q)VV @VV\pi_1(p)V \\
\pi_1(U,x) @>>\pi_1(\iota)> \pi_1(X,x) \\
\end{CD}
\]
$\tilde X$ ist einfach zusammenhängend, also ist $\pi_1(\tilde X,y)=\{1\}$ und man sieht im Diagramm: $\pi_1(\iota): \pi_1(U,x)\to\pi_1(X,x)$ ist der triviale Homomorphismus, das heißt jeder geschlossene Weg in $U$ ist nullhomotop in $X$, und damit ist $X$ semi-lokal einfach zusammenhängend. 


\paragraph{Behauptung:} $X$ hat keine universelle Überlagerung.

$X$ ist nicht semi-lokal einfach zusammenhängend, denn der Punkt $(0,0)$ hat keine Umgebung, in der jeder geschlossene Weg nullhomotop ist, da in jeder Umgebung von $(0,0)$ einen Kreis enthält.


\section*{Aufgabe 3}

Seien $p_1:Y_1\to X$ und $p_2:Y_2\to X$ Überlagerungen und $Y_1,Y_2$ zusammenhängend.

\paragraph{Behauptung:} Ein Morphismus $f:Y_1\to Y_2$ (d.h. eine stetige Abbildung $f:Y_1\to Y_2$ mit $p_1=p_2\circ f$) ist eine Überlagerung.

Sei $y\in Y_2$ und $x\da p_2(y)$. Dann gilt: $\tilde y\in f^{-1}(y)\implies f(\tilde y)=y \implies (p_2\circ f)(\tilde y) = p_2(y) \implies p_1(\tilde y) = p_2(y)$. $p_1$ und $p_2$ sind Überlagerungen, also gibt es eine Umgebungen $U\subseteq X$ und $\tilde U\subseteq X$ von $x$, so dass $p_1^{-1}(U)=\bigcup_{i\in I}V_i$ und $p_2^{-1}(\tilde U) = \bigcup_{j\in K}\tilde V_j$, so dass die $V_i$ und $\tilde V_j$ offen, untereinander paarweise disjunkt und zusammenhängend sind. Sei o.B.d.A $U=\tilde U$.

Ist $f$ surjektiv, so ist $f^{-1}(y)\ne 0$. Es gibt ein $j\in J$, so dass $y\in \tilde V_j$. Dann ist $f^{-1}(\tilde V_j) = \bigcup_{i\in \tilde I\subseteq I} V_i$, woraus die Behauptung folgt.

\section*{Aufgabe 4}

\paragraph{Behauptung:} Für $n\ge 2$ gilt: $\pi_1(\mathbb P^n(\mathbb R)) = \mathbb Z/2\mathbb Z$.

Definiere Operation von $\mathbb Z/2\mathbb Z$ auf $S^n$ durch $\bar 0 \cdot x = x$, $\bar 1\cdot x = -x$. Diese Operation ist eigentlich diskontinuierlich und fixpunktfrei, also ist $\pi: S^n \to S^n/(\mathbb Z/2\mathbb Z)$ ist eine Überlagerung, und $S^n/(\mathbb Z/2\mathbb Z)$ ist gerade $\mathbb P^n(\mathbb R)$. Für $n\ge 2$ ist $S^n$ einfach zusammenhängend, also ist $\pi$ eine universelle Überlagerung und $\pi_1(\mathbb P^n(\mathbb R)) \cong \operatorname{Deck}(\pi) \cong \mathbb Z/2\mathbb Z$.

\end{document}
