\documentclass{article}
\usepackage[utf8]{inputenc}
\usepackage{amsmath}
\usepackage{amsfonts}
\usepackage{amssymb}
\usepackage{amsthm}
\usepackage{mathrsfs}
\usepackage{german}
\usepackage{enumerate}
\usepackage{stmaryrd}
\title{4. Topologie Übung}
\author{Ferdinand Szekeresch}
\begin{document}
\maketitle

\textbf{Aufgabe 2}
\begin{enumerate}[a)]
\item $\mathbb{R}, (a,b), [a,b], (a,b]$\begin{itemize}
\item $\mathbb{R}$ und $(a,b)$ sind homöomorph, denn :\\
$$f_1 : \mathbb{R}\rightarrow (0,1); x\mapsto\frac{1}{\pi}\arctan(x)+\frac12 \text{ ist homöomorph}$$
$$f_2 : (0,1) \rightarrow (a,b), x\mapsto(b-a)x+a \text{ ist homöomorph}$$
$\Rightarrow f_2\circ f_1: \mathbb{R}\rightarrow(a,b)$ ist Homöomorphismus.
\item $(a,b)$ und $[a,b]$ sind nicht homöomorph, denn $(a,b)$ ist nicht kompakt, $[a,b]$ aber schon. Da stetige Abbildungen Kompakta auf Kompakta abbilden und Homöom. insbes. stetig sind, kann es keinen Homöomorphismus $(a,b)\rightarrow[a,b]$ geben.
\item $[a,b]\rightarrow(a,b]$ sind nicht homöomorph, wäre $f:[a,b]\rightarrow(a,b]$ ein Homöomorphismus, so wäre $f$ nach Zwischenwertsatz streng monoton, d.h. $f([a,b]) = [f(a),f(b)]\lightning$.
\item analog: $(a,b)$ und $(a,b]$ sind nicht homöom.\\
$\Rightarrow \mathbb{R}$ und $(a,b)$ bzw. $(a,b]$ sind nicht homöom.
\end{itemize}
\item $S^1$ und $\mathbb{R}/\mathbb{Z}$ sind homöom.\\
Definiere Homöomorphismus $h:\mathbb{R}/\mathbb{Z}\rightarrow S^1, [x]\mapsto \big(\cos(2\pi x),\sin(2\pi x)\big)$\\
$h$ ist wohldefiniert, denn seien $x,y$ mit $x\sim y \Leftrightarrow\exists k\in\mathbb{Z}:x=y+k$\\
$\Leftrightarrow h([x]) = \big(\cos(2\pi y + 2\pi k),\sin(2\pi y + 2\pi k)\big) = \big(\cos(2\pi y),\sin(2\pi y)\big) = h([y])$\\
Die zeigt auch: $h$ ist injektiv.\\
Klar: $h$ ist surjektiv.\\
$h$ ist stetig, da $h\circ\pi$ stetig ist, ($+$ Aufgabe 4, Blatt 5)\\
$h$ ist offen, denn $h\circ \pi$ ist offen.\\
Das reicht, denn $\forall O\subseteq\mathbb{R}/\mathbb{Z}: h(o) = h\circ\pi\big(\pi^{-1}(O)\big)$, da $\pi$ surjektiv ist.)\\
Das überlegt man sich für Intervalle $\subseteq\mathbb{R}$.
\pagebreak
\item $W^n:= \partial\big([0,1]^{n+1}\big), S^n :=\{x\in\mathbb{R}^{n+1}|\|x\|=1\}$ sind homöomorph.\\
Definiere $$f:S^n\rightarrow W^n, (x_1,\ldots,x_{n+1})\mapsto\frac{1}{\max(|x_1|,\ldots,|x_{n+1})}(x_1,\ldots,x_{n+1})$$
$$g:S^n\rightarrow W^n, (y_1,\ldots,y_{n+1})\mapsto\frac{1}{\max(|y_1|,\ldots,|y_{n+1})}(y_1,\ldots,y_{n+1})$$
$f$ und $g$ sind stetig zueinander.
\end{enumerate}

\textbf{Aufgabe 1}
\begin{enumerate}[a)]
\item $\mathbb{Q}$ ist abzählbar $\stackrel{\text{b)}}{\Rightarrow} \big\{\{x\}|x\in\mathbb{Q}\big\}$ ist die Menge der Zusammenhangskomponenten von $\mathbb{Q}$.
\item Beh: $(X,d)$ abzählbar $\Rightarrow$ die Zusammenhangskomponenten von $X$ sind einelementig.\\
Bew: Seien $x\neq y\in X\Rightarrow l:=d(x,y)>0$\\
$X$ abzählbar $\Rightarrow l\in M$, wobei $M$ abzählbare Teilmenge von $\mathbb{R}$\\
$\Rightarrow\exists r\in[0,d]:\{z\in X|d(x,z) = r\} = \emptyset$\\
Setze $V_1 = \{z\in X|d(x,z) \leq r\} V_2 = \{z\in X|d(x,z) \geq r\}$\\
Gäbe es eine zusammenhängende Teilmenge $A$ von $X$ mit $x,y\in A$, so wäre $A=\underbrace{(V_1\cap A)}_{\neq\emptyset}\cup\underbrace{(V_2\cap A)}_{\neq\emptyset}\quad\lightning$ zu $A$ zusammenhängend.
\end{enumerate}

\textbf{Aufgabe 3}\\
Seien jetzt aber $A\subseteq B\subseteq A$ mit $A$ zusammenhängend und $U,V$ disjunkte offene Teilmengen mit $B=U\cup V$\\
$\Rightarrow\underbrace{(U\cap A)}_{=:\tilde U}\cup\underbrace{(V\cap A)}_{=:\tilde{V}} = (U\cup V)\cap A = A$\\
$\tilde{U},\tilde{V}$ sind disjunkt (wegen $\tilde{U}\cap\tilde{V}\subseteq U\cap V = \emptyset$)\\
$A$ zusammenhängend $\Rightarrow \tilde{U}=\emptyset$ oder $\tilde{V}=\emptyset$. O.B.d.A $\tilde{U}=\emptyset\Rightarrow A\subseteq V$\\
$\Rightarrow U\subseteq B\subseteq \bar{A}\subseteq\bar{V}\Rightarrow U=U\cap\bar{V} = \emptyset\Rightarrow B$ ist zusammenhängend.

\textbf{Aufgabe 4}\\
Beh. $X$ top., $A\subseteq X, Y$ Hausdorffraum, $f: A\rightarrow Y$ stetige Abbildung.\\
$\Rightarrow$ kann man $f$ fortsetzen zu einer stetigen Abb. $g:\bar{A}\rightarrow Y$, so ist $g$ eindeutig.\\
Bew: Seien $g_1: \bar{A}\rightarrow Y, g_2:\bar{A}\rightarrow Y$ stetige Fortsetzungen von $A$.\\
Ann: $g_1\neq g_2 \Rightarrow x\in\bar{A}:g_1(x)\neq g_2(x)$. Es muss gelten: $x\notin A$. da $\forall x\in A: g_1(x)=f(x)=g_2(x)$.\\
Also: $x\in \bar{A}\backslash A$.\\
$Y$ Hausdorffraum, $g_1(x)\neq g_2(x)\Rightarrow\exists$ offene disj. Teilmengen $V_1, V_2\subseteq Y$ mit $g_1(x)\in V_1, g_2(x)\in V_2$.\\
$g_1$ ist stetig $\Rightarrow\exists$ offene Umg. $U_1$ von $x$ mit $g_1(U_1)\subseteq V_1$\\
$g_2$ ist stetig $\Rightarrow\exists$ offene Umg. $U_2$ von $x$ mit $g_2(U_2)\subseteq V_2$\\
$U_1, U_2$ sind offene Umgebungen von $x \Rightarrow U_1\cap U_2$ ist offene Umg. von $x \Rightarrow x\in U_1\cap U_2$\\
$x\in\partial A \Rightarrow\exists y\in(U_1\cap U_2)\cap A$ mit $x\neq y$ (nach Aufg. 2, Blatt 3)\\
$\Rightarrow y\in U_1\Rightarrow g_1(y)\in V_1, y\in U_2\Rightarrow g_2(y)\in V_2$\\
Da aber $y \in A$ gilt: $g_1(y)=f(y)=g_2(y)$\\
$\Rightarrow f(y)\in V_1\cap V_2\quad\lightning$ zu $V_1\cap V_2\neq\emptyset$.
\end{document}
