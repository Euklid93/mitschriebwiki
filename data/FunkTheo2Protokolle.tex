\documentclass[11pt]{article}

\usepackage{latexki}

\lecturer{Dr. Herzog}
\title{Funktionentheorie 2 -- Protokolle}
\semester{Wintersemester 09/10}
\scriptstate{partial}


\usepackage[T1]{fontenc}
\usepackage[utf8]{inputenc}
\usepackage{ae} 
\usepackage[english,ngerman]{babel} 
\usepackage{epsf, amsmath, amssymb, graphicx, epsfig, hyperref, amsthm}
\usepackage{graphicx}
\usepackage{amsmath}
\usepackage{geometry}
\geometry{a4paper,tmargin=25mm,bmargin=30mm,lmargin=15mm,rmargin=20mm}
\usepackage{bbold}
\usepackage{color}

\setlength{\parindent}{0pt} %Setzt den Erstzeileneinzug auf Null.
\setlength{\parskip}{\baselineskip} %Trennt Absätze durch eine zusätzliche Leerzeile.

\newcommand{\C}{\mathbb{C}}
\newcommand{\K}{\mathbb{K}} 
\newcommand{\R}{\mathbb{R}}
\newcommand{\Q}{\mathbb{Q}} 
\newcommand{\Z}{\mathbb{Z}} 
\newcommand{\N}{\mathbb{N}}
\newcommand{\D}{\mathbb{D}}
\newcommand{\T}{\mathbb{T}}

%\vspace{-0.5cm}
%\begin{itemize}
%\item 
%\end{itemize}

\begin{document}

\begin{center}
{\Large \textbf{Funktionentheorie I - Prüfungsprotokolle}} \\
\textit{Skript Dr. Herzog}
\end{center}

\underline{§ 1 Komplexe Zahlen}

\underline{§ 2 Topologische Begriffe}
\vspace{-0.5cm}
\begin{itemize}
\item Offene Menge
\item Abgeschlossene Menge
\item Kompakte Teilmenge: Überdeckungskompaktheit
\end{itemize}

\underline{§ 3 Die Riemann'sche Zahlenkugel}

\underline{§ 4 Konvergenz und Stetigkeit in $\hat{\C}$} 

\underline{§ 5 Komplex differenzierbare Funktionen}
\vspace{-0.5cm}
\begin{itemize}
\item Definition: Holomorphe Funktion ($\rightarrow$ In jedem Punkt einer offenen Menge komplex differenzierbar)
\item Beispiele für holomorphe Funktionen: \\
\quad Potenzreihen \\
\quad Konstante Funktionen \\
\quad $f(z) = z$ \\
\quad Polynome \\
\quad $f(z) = \overline{z}$ ist in keiner Stelle komplex differenzierbar ($\rightarrow$ CRD)
\item Komplexe Differenzierbarkeit ($\rightarrow$ reelle Differenzierbarkeit + CRD)
\item Sind alle holomorphen Funktionen stetig?
\end{itemize}

\underline{§ 6 Potenzreihen}
\vspace{-0.5cm}
\begin{itemize}
\item Was ist eine Potenzreihe? Wie sieht ihr Konvergenzverhalten aus?
\item Satz von Cauchy-Hadamard \\
\quad Was gilt für den Rand des Konvergenzkreises einer Potenzreihe? \\
\quad Was weiß man über das Konvergenzverhalten?
\item Satz 6.2
\item Satz von Abel
\item Konvergenzradius der geometrischen Reihe
\item Funktion auf $\D$ gegeben: Ist die Funktion auf $\D$ holomorph?
\end{itemize}

\underline{§ 7 Stammfunktionen}

\newpage
\underline{§ 8 Die Funktionen $e^z$, $\log z$, $z^q$}
\vspace{-0.5cm}
\begin{itemize}
\item Definition Logarithmus ($\rightarrow$ $e^z = w$ $\Rightarrow$ $\log(w) = \log|w| + i\arg(w)$)\\
\quad Wo ist dieser definiert? ($\rightarrow$ auf $\C \backslash (-\infty, 0]$) \\
\quad Warum? ($\rightarrow$ Umkehrfunktion der Exponentialfunktion)
\end{itemize}

\underline{§ 9 Möbiustransformationen}
\vspace{-0.5cm}
\begin{itemize}
\item Definition Möbiustransformation \\
\quad Möbiustransformationen bilden verallgemeinerte Kreise auf verallgemeinerte Kreise ab
\item Wie transformiert man die Einheitskreisscheibe auf die obere Halbebene in $\C$? ($\rightarrow$ $\Phi(z) = \frac{-iz+1}{z-i}$)
\end{itemize}

\underline{§ 10 Das Wegintegral}
\vspace{-0.5cm}
\begin{itemize}
\item Definition (geschlossener) Weg
\item Definition Wegintegral
\item Was bedeutet, dass eine holomorphe Funktion lokal in eine Potenzreihe entwickelbar ist? \\
\quad Wie groß ist der Konvergenzradius mindestens?
\item Definition Umlaufzahl \\
\quad Wo ist die Umlaufzahl definiert? ($\rightarrow$ auf $\C \backslash \gamma*$) \\
\quad Umlaufzahl anhand eines Beispiels bestimmen
\end{itemize}

\underline{§ 11 Der lokale Cauchy'sche Integralsatz}
\vspace{-0.5cm}
\begin{itemize}
\item Hat jede holomorphe Funktion eine Stammfunktion? \\
\quad Beispiel für eine Funktion, die keine Stammfunktion hat ($\rightarrow$ $f(z) = \frac{1}{z}$) \\
\quad Wie sieht es aus, wenn man den Definitionsbereich auf $\C \backslash (-\infty,0]$ einschränkt? ($\rightarrow$ $\log(z)$ ist dann die Stammfunktion)
\item Lokaler CIS (+ Beweis)
\item CIS für Dreiecke/Lemma von Goursat (+ Beweisidee) \\
\quad Brauchen wir da nicht mehr als $\Omega$ offen? ($\rightarrow$ Nein, weil wir $\Delta \subseteq \Omega$ fordern)
\item CIS für konvexe Mengen (+ Beweis)\\
\quad Wie sieht die Stammfunktion von $f$ aus? Wie wird die Stammfunktion konstruiert?  \\
\quad Wo gibt es eine Stammfunktion? \\
\quad Warum ist das Integral über den Weg Null? \\
\quad Warum gibt es den Ausnahmepunkt im CIS? Warum ist er künstlich? ($\rightarrow$ Siehe Beweis der CIF, Riemann'scher Hebbarkeitssatz) \\
\quad $\int_{\gamma} \frac{\sin^2(z)}{z^2 (z-2)} \, dz$ berechnen für $\gamma(t) = e^{it}$ ($\rightarrow$ Nach CIS $=0$, $z=0$ ist hebbar, $z=2$ liegt außerhalb des Wegs) \\
\quad Was folgt aus dem CIS? ($\rightarrow$ CIF) \\
\quad Verallgemeinerungen
\item CIF für konvexe Mengen (+ Beweis) \\
\quad Daraus folgt die lokale Entwickelbarkeit der Funktion als Potenzreihe
\item Folgerung für holomorphe Funktionen: \\
\quad Potenzreihenentwicklung ($\rightarrow$ $f(z) = \sum_{n=0}^{\infty}) a_n (z-z_0)^n$) \\
\quad Wie berechnet man die Koeffizienten? ($\rightarrow$ $a_n = \frac{f^{(n)}(z_0)}{n!}$)\\
\quad Konvergenzradius kann größer als $R$ sein \\
\quad Entwicklungssatz (+ Beweis, Bild)\\
\quad Warum kann man eine holomorphe Funktion als Potenzreihe darstellen und wo? ($\rightarrow$ Konvergenzradius) \\
\quad Was bedeutet die Aussage, dass eine holomorphe Funktion lokal durch eine Potenzreihe darstellbar ist? \\
\quad $f(z) = \frac{1}{\cos(z)}$: Potenzreihe um $i$, wie groß ist der Konvergenzradius? \\
\quad $f(z) = \frac{1}{\tan(z)}$ \\
\quad $f(z) = \frac{1}{\sin(z)}$ \\
\quad Konvergenzradius von $f(z) = \tan(z)$ ($\rightarrow R=\frac{\pi}{2}$) \\
\quad Konvergenzradius mit Hilfe des Satzes von Pythagoras berechnen \\
\quad Potenzreihenentwicklung der geometrischen Reihe
\end{itemize}

\underline{§ 12 Eigenschaften holomorpher Funktionen}
\vspace{-0.5cm}
\begin{itemize}
\item Nullstellensatz ($\rightarrow$ Keine Häufungspunkte)\\
\quad Fragen zur Menge $Z(f)$
\item Identitätssatz \\
\quad $f(\frac{1}{n}) = 0 \, \forall n \in \N$: Wie könnte $f$ heißen? ($\rightarrow$ $f(z) = 0$) \\
\quad Gibt es noch andere Lösungen? ($\rightarrow$ Nach dem Identitätssatz gibt es keine weiteren Lösungen) \\
\quad Beweis für den Fall $f=0$ ($\rightarrow$ $Z(f)$ hat HP in $\Omega$ $\Rightarrow$ $f = 0$ auf $\Omega$)
\item Isolierte Singularität \\
\quad Nur die drei? ($\rightarrow$ Verweis auf Beweis)
\item Riemann'scher Hebbarkeitssatz (+ Beweis)
\item Klassifikation isolierter Singularitäten ($\rightarrow$ Eigene Beispiele parat haben) \\
\quad Singularitäten bestimmen: $f(z) = \frac{1}{\sin(\frac{1}{z})}$, $f(z) = e^{\frac{1}{z}}$, $f(z) = \frac{1}{e^z + 1}$, $f(z) = \frac{1}{1-e^{\frac{1}{z}}}$ 
\item Pol \\
\quad Was macht eine Funktion bei einem Pol?
\item Wesentliche Singularität \\
\quad Beispiel für Funktion mit wesentlicher Singularität ($\rightarrow$ $f(z) = e^{\frac{1}{z}}$, wesentliche Singularität bei $0$)\\
\quad Satz von Casorati-Weierstraß \\
\quad Was bedeutet \glqq Bild liegt dicht in $\C$\grqq{}? \\
\quad Was passiert mit der Funktion in einer wesentlichen Singularität?
\item Cauchy'sche Integralformel für Ableitungen
\item Cauchy'sche Ungleichung (+ Beweis)
\item Satz von Liouville (+ Beweis)
\item Hauptsatz der Algebra
\item Satz von der Gebietstreue (+ Beweis)
\item Maximumprinzip (+ Beweis) \\
\quad Minimumprinzip ($\rightarrow$ $f(z) \neq 0$) \\
\quad Warum Gebiet wichtig? ($\rightarrow$ Wegen zusammenhängend) \\
\quad Kann man eine Funktion finden, die Maximum/Minimum annimmt? ($\rightarrow$ Ja, $f$ konstant)
\item Schwarz'sches Lemma (+ Beweis) ($\rightarrow$ Anwendungen: Automorphismen, Subordination)
\end{itemize}

\underline{§ 13 Das lokale Abbildungsverhalten}

\underline{§ 14 Folgen holomorpher Funktionen}
\vspace{-0.5cm}
\begin{itemize}
\item Kompakte Konvergenz
\item Konvergenzsatz von Weierstraß (+ Beweis) \\
\quad Woher kommt die Konvergenz der Ableitungen? ($\rightarrow$ CIF für Ableitungen) \\
\quad Was gilt, wenn wir eine holomorphe Folge haben ($\rightarrow$ Konvergenzsatz von Weierstraß)
\end{itemize}

\underline{§ 15 Der globale Cauchy'sche Integralsatz}
\vspace{-0.5cm}
\begin{itemize}
\item Definition Zykel
\item Globaler CIS \\
\quad Wie kann man CIS und CIF verallgemeinern? \\
\quad Was ändert sich an den Voraussetzungen? \\
\quad Wo muss der Index Null sein?
\item Unterschied zwischen dem globalem CIS und CIS für konvexe Mengen ($\rightarrow$ Indexbedingung) \\
\quad Beispiel, für das die Indexbedingung gilt
\end{itemize}

\underline{§ 16 Einfach zusammenhängende Gebiete}
\vspace{-0.5cm}
\begin{itemize}
\item Homotopie
\item Nullhomotopie
\item Einfach zusammenhängendes Gebiet \\
\quad Darf es in einem ezsh. Gebiet Löcher geben? ($\rightarrow$ Nein)\\
\quad Beispiel für Gebiete, in denen $ind_\gamma (\alpha) = 0$ für alle $\alpha \in \C \backslash \Omega$ immer erfüllt ist ($\rightarrow$ ezsh. Gebiete)\\
\quad Einfach zusammenhängendes, nicht konvexes Gebiet zeichnen \\
\quad Nicht einfach zusammenhängendes Gebiet zeichnen \\
\quad Offene, aber nicht zsh. Menge zeichnen
\item Was gilt auf einfach zusammenhängenden Gebieten? ($\rightarrow$ $ind_\gamma(\alpha) = 0$ für $\alpha \in \C \backslash \Omega$)
\item Holomorpher Logarithmus (+ Beweis)
\end{itemize}

\underline{§ 17 Der Residuensatz}
\vspace{-0.5cm}
\begin{itemize}
\item Meromorphe Funktion
\item Residuum
\item Residuensatz (+ Beweisidee) \\
\quad Endlichkeit der Summe, Häufungspunkte ($\rightarrow$ Definition meromorphe Funktion: Es darf Häufungspunkte geben, aber nur auf dem Rand. Da der Weg komplett innerhalb des Gebiets liegt, umläuft man auf keinen Fall irgendwelche Häufungspunkte und damit ist die Summe endlich, Menge der Pole abzählbar unendlich oder endlich) \\
\quad Anwendungen: Argumentprinzip, Satz von Rouché, Berechnung uneigentlicher Integrale
\item Argumentprinzip (+ Beweis)
\item Satz von Rouché \\
\quad Welche Nullstellen sind gemeint? ($\rightarrow$ Im Inneren des geschlossenen Wegs)\\
\quad Anwendungen des Satzes von Rouché ($\rightarrow$ Satz von Hurwitz)
\item Berechnung uneigentlicher Integrale (+ Beweis) \\
\quad $\int_{-\infty}^{\infty} \frac{1}{1+x^4}$: Wo liegen die Polstellen? ($\rightarrow$ Über die Residuen in der oberen Halbebene summieren) \\
\quad Warum werden da gerade die Nullstellen der Funktion gezählt? ($\rightarrow$ Residuensatz) \\
\quad Warum muss $grad \, Q \geq grad \, P +2$ gelten und $Q(x) \neq 0$ im Reellen? ($\rightarrow$ Im Reellen gibt es Polynome vom Grad $n > 0$, die keine Nullstellen besitzen, im Komplexen wegen des Fundamentalsatzes der Algebra nicht)

\item Satz von Rouché (im Zusammenhang mit Satz von Hurwitz) \\
\quad Rechenbeispiel
\end{itemize}

\underline{§ 18 Laurent-Reihen}
\vspace{-0.5cm}
\begin{itemize}
\item Entwicklungssatz von Laurent \\
\quad Konvergenzverhalten \\
\quad Ringgebiete
\end{itemize}

\underline{§ 19 Eine Anwendung des Schwarz'schen Lemmas}
\vspace{-0.5cm}
\begin{itemize}
\item Blaschkefaktor ($\rightarrow$ Anwendung des Schwarz'schen Lemmas)
\item Automorphismen des Einheitskreises 
\end{itemize}

\underline{§ 20 Harmonische Funktionen}

\newpage

\begin{center}
{\Large \textbf{Funktionentheorie II - Prüfungsprotokolle}} \\
\textit{Skript Dr. Herzog}
\end{center}

\underline{§ 1 Schreibweisen und Wiederholung}

\underline{§ 2 Der Satz von Montel}
\vspace{-0.5cm}
\begin{itemize}
\item Satz von Arzelà-Ascoli
\item Satz von Montel \\
\quad lokal gleichmäßig beschränkt
\item Normale Familie
\item Was heißt (lokal) gleichmäßige Konvergenz?
\end{itemize}

\underline{§ 3 Der Riemann'sche Abbildungssatz}
\vspace{-0.5cm}
\begin{itemize}
\item Konforme Abbildung
\item Riemann'scher Abbildungssatz (+ Beweisidee) \\
\quad Welcher Satz geht in den Beweis ein? ($\rightarrow$ Satz von Montel)\\
\quad Wenn wir ein offenes zusammenhängendes Gebiet haben, welches nicht die ganze Ebene ist, was können wir dann über den Automorphismus von $\D$ sagen?
\end{itemize}

\underline{§ 4 Automorphismen spezieller Gebiete} 
\vspace{-0.5cm}
\begin{itemize}
\item Automorphismus von $\D$ (+ Beweis)
\item Blaschkefaktor \\
\quad Gibt es noch andere Automorphismen in $\D$ außer den Blaschkefaktor ($\rightarrow$ Nein. Siehe Beweis)
\item Automorphismus von $\C$ (+ Beweis) \\
\quad Warum ist f ein Polynom? ($\rightarrow$ Laurententwicklung von $g$ hat unendlich viele Terme $\Rightarrow$ $f$ ist kein Polynom) \\
\quad Wieso sind das alle Automorphismen und wieso sind das überhaupt Automorphismen? ($\rightarrow$ Siehe Beweis)
\item Lemma 4.3 (+ Beweis)
\item Aut(G)
\item Wie transformiert man die rechte Halbebene in $\C$ auf die Einheitskreisscheibe? ($\rightarrow$ Erst quadrieren, dann $\Phi(z) = \frac{z-i}{z+i}$)
\item Was bildet die offene Kreisscheibe auf die offene Kreisscheibe ab? ($\rightarrow$ Blaschkefaktor)
\end{itemize}

\underline{§ 5 Harmonische Funktionen}
\vspace{-0.5cm}
\begin{itemize}
\item Harmonische Funktionen \\
\quad Beipiel für harmonische Funktion: Realteil einer holomorphen Funktion \\
\quad Beipiel für eine harmonische Funktion, die kein Realteil einer holomorphen Funktion ist: $\log|z|$
\item Satz 5.1 (+ Beweis) \\
\quad Was ist wenn $f$ holomorph ist auf einem Gebiet und dort keine Nullstellen besitzt? ($\rightarrow$ Dann ist $\log|f|$ harmonisch)
\item Ist jede harmonische Funktion Realteil von einer holomorphen Funktion? ($\rightarrow$ Auf einfach zusammenhängenden Gebieten ja)
\item Konjugiert harmonische Funktion \\
\quad Was gilt alles: MWE, Maximum/Minimum, Identitätssatz
\item MWE (+ Beweis)
\item Poisson'sche Integralformel
\end{itemize}

\underline{§ 6 Konforme Äquivalenz von Ringgebieten}
\begin{itemize}
\item Aut(A) (A ist Ringgebiet um $0$)
\item Ziel
\end{itemize}

\underline{§ 7 Das Schwarz'sche Spiegelungsprinzip}


\underline{§ 8 Der Satz von Bloch}
\vspace{-0.5cm}
\begin{itemize}
\item Satz von Bloch \\
\quad Folgerungen: Kreisscheiben
\end{itemize}

\underline{§ 9 Der kleine Satz von Picard}
\vspace{-0.5cm}
\begin{itemize}
\item Kleiner Satz von Picard (+ Beweis über großen Satz von Picard) \\
\quad Gibt es holomorphe Funktionen, die einen Wert aus $\C$ weglassen? ($\rightarrow$ Ja, $\exp(\C) = \C \backslash \{0\}$)
\item Kleiner Satz von Picard für meromorphe Funktionen (+ Beweis) \\
\quad Beispiel $f(z) = \frac{1}{1 + e^z}$, $0,1 \notin f(\C)$ \\
\quad Warum gilt $g \in H(\C)$? Was passiert mit den Polen von $f$? \\
\quad Was passiert mit den Polen?
\item Anwendung des kleinen Satzes von Picard: Fixpunkt bei periodischer Funktion (+ Beweis)
\item Beispiel anhand des Tangens $f(z) = \tan z$ ($\rightarrow$ Nimmt $i$ und $-i$ nicht an) \\
\quad Beweis über die Ableitung des Tangens: $1+\tan^2(z) = \frac{1}{\cos^2(z)}$ ist lösbar
\item Welche Anwendungen gibt es? \\
\quad Sätze von Iyer \\
\quad Beispiel: $\sin^2(z) + \cos^2(z) = 1$ \\
\quad In welchem berühmten Satz wird das angewandt? ($\rightarrow$ Satz von Fermat-Wiles)
\end{itemize}

\underline{§ 10 Schlichte Funktionen in $\D$}
\vspace{-0.5cm}
\begin{itemize}
\item Definition der Menge $S$
\item Potenzreihendarstellung
\item Koebefunktion 
\item Satz von Bieberbach
\item Bieberbach'sche Vermutung \\
\quad Was kann man über die Koeffizienten sagen? ($\rightarrow$ $|a_n| \leq n$)
\item Koebe'scher $\frac{1}{4}$-Satz 
\end{itemize}

\underline{§ 11 Zur Potenzreihendarstellung holomorpher Funktionen}
\vspace{-0.5cm}
\begin{itemize}
\item Singulärer Punkt, regulärer Punkt (auf dem Rand einer Kreisscheibe) \\
\quad Beispiele \\
\quad Welche Punkte gibt es immer? ($\rightarrow$ Singuläre Punkte) \\
\quad Hat jede holomorphe Funktion auf $\D$ singuläre Punkte? \\
\quad Beispiel, dass es keine regulären Punkte auf $\D$ geben muss? ($\rightarrow \sum_{n=0}^\infty z^{n!}$) \\
\quad Wieso ist die Menge der singulären Punkte abgeschlossen?
\item Satz über die Existenz singulärer Punkte (+ Beweis)
\item Satz von Pringsheim \\
\quad Singuläre Punkte von $\sum_{n=0}^\infty \frac{1}{n^2}z^n$
\end{itemize}

\underline{§ 12 Der Satz von Mittag-Leffler}
\vspace{-0.5cm}
\begin{itemize}
\item Satz von Mittag-Leffler \\
\quad Wie sieht $f$ aus? \\
\quad Wie wählt man $p_n$? \\
\quad Wie wählt man $q_n$? 
\end{itemize}

\underline{§ 14 Unendliche Produkte}
\vspace{-0.5cm}
\begin{itemize}
\item Unendliches Produkt \\
\quad Wann ist es konvergent?
\end{itemize}

\underline{§ 15 Der Weierstraß'sche Produktsatz}
\vspace{-0.5cm}
\begin{itemize}
\item Produktsatz von Weierstraß \\
\quad Definition \\
\quad Bedeutung \\
\quad Was sind $E_{p_n}$? \\
\quad Wie werden die $p_n$ gewählt? \\
\quad Kann man immer ein $p_n$ finden? \\
\quad Wie sieht die Funktion $f$ aus und welche Eigenschaften hat sie? \\
\quad Beispiel: Eine auf $\C$ holomorphe Funktion angeben, die in den natürlichen Zahlen Nullstellen hat \\
\quad Beispiel: Funktion konstruieren, die in natürlichen Zahlen doppelte Nullstellen besitzt \\
\quad Was tut man, wenn man die $0$ als Nullstelle der Ordnung $5$ haben möchte? ($\rightarrow$ Weierstraßprodukt mit $z^5$ multiplizieren)
\item Was heißt lokal gleichmäßige Konvergenz? ($\rightarrow$ Kompakte Konvergenz)
\item Folgerung aus dem Produktsatz von Weierstraß: Zusammenhang zwischen meromorphen und holomorphen Funktionen (+ Beweis)
\end{itemize}

\underline{§ 16 Der Ring $H(\C)$}
\vspace{-0.5cm}
\begin{itemize}
\item Definition: Ideal, endlich erzeugt, Hauptideal, Hauptidealring
\item Ist der Ring der holomorphen Funktionen auf einem Gebiet immer nullteilerfrei?
\item $H(\C)$ ist kein Hauptidealring (+ Beweis) \\
\quad Welches Ideal benötigt man, um das zu zeigen? \\
\quad Beispiel für ein Ideal im Ring $H(\C)$, das nicht endlich ist erzeugt ist \\
\quad $H(\C)$ ist kein Hauptidealring. Was gilt aber? ($\rightarrow$ Jedes endlich erzeugte Ideal ist Hauptideal)
\end{itemize}

\underline{§ 17 Die Jensen'sche Formel}
\vspace{-0.5cm}
\begin{itemize}
\item Definition Jensen'sche Formel + Bedeutung
\end{itemize}

\underline{§ 18 Periodische Funktionen}
\vspace{-0.5cm}
\begin{itemize}
\item Was sind Perioden?
\end{itemize}

\underline{§ 19 Elliptische Funktionen}
\vspace{-0.5cm}
\begin{itemize}
\item Elliptische Funktion
\item 1. Satz von Liouville
\item 2. Satz von Liouville
\item 3. Satz von Liouville
\item Weierstraß'sche $\rho$-Funktion
\item Die Differentialgleichung der Weierstraß’schen $\rho$-Funktion \\
Satz über die Darstellung aller elliptischen Funktionen durch 2 rationale Funktionen
\end{itemize}

\underline{§ 20 Der Fixpunktsatz von Earle-Hamilton in $\C$}
\vspace{-0.5cm}
\begin{itemize}
\item Strikte Teilmenge
\item Fixpunktsatz von Earle-Hamilton
\end{itemize}

\underline{§ 21 Die Subordination}
\vspace{-0.5cm}
\begin{itemize}
\item Prinzip der Subordination (+ Beweis)
\end{itemize}

\underline{§ 22 Verbindungen zur Funktionalanalysis und die Sätze von Montel und Vitali}

\end{document}
